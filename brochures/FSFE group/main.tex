\documentclass[10pt,foldmark,tumble]{leaflet}
\usepackage[utf8]{inputenc}
\usepackage{mathptmx}
\usepackage{url}
\usepackage{graphicx,txfonts}
\usepackage[dvipsnames,usenames]{color}
\usepackage{dirtytalk}

\usepackage[framemethod=TikZ]{mdframed}
\mdfdefinestyle{MyFrame}{%
    linecolor=blue,
    outerlinewidth=2pt,
    roundcorner=20pt,
    innertopmargin=\baselineskip,
    innerbottommargin=\baselineskip,
    innerrightmargin=20pt,
    innerleftmargin=20pt,
    backgroundcolor=gray!30!white}

\newcommand{\love}{\ensuremath\varheartsuit\ }

\CutLine*{1}% Dotted line without scissors
\CutLine*{6}
%%\CutLine{6}%  Dotted line with scissors

%%\AddToBackground{6}{%  Background of a small page
%%  \put(0,0){\textcolor{Cyan}{\rule{\paperwidth}{\paperheight}}}}


\begin{document}

\begin{flushright}
\includegraphics[scale=0.8]{gFSFE.png}
\end{flushright}

\includegraphics[scale=0.20]{fsfe.pdf}

\section{\includegraphics{item.png}La misión de la FSFE}

El Software envuelve todos los aspectos de nuestras vidas y es importante que esta tecnología nos potencie en vez de restringirnos. El Software Libre da a todos los derechos de usar, estudiar, adaptar y compartir software. Estes derechos ayudan a sustentar otros derechos fundamentales como la libertad de expresión, libertad de prensa y privacidad.

La Fundación del Software Libre de Europa:
\begin{itemize}
\item ayuda a individuos y organizaciones a comprender cómo el Software Libre contribuye a la libertad, transparencia y autodeterminación.
\item mejora los derechos de los usuarios eliminando las barreras para la adopción del Software Libre.
\item Anima a personas a usar y desarrollar Software Libre.
\item Proporciona recursos para permitir a todos promover el Software Libre en Europa.
\end{itemize}


\section{\includegraphics{item.png}Acerca de la FSFE}

  \subsection{Concepto propio}

Las personas de la Fundación del Software Libre de Europa (FSFE) nos vemos como europeos de diferentes culturas con el objetivo común de cooperar entre culturas y desarrollar una cultura común de cooperación desde un nivel regional hasta uno global.

Formamos una organización no gubernamental sin ánimo de lucro y una red que en sí misma es parte de una red global de personas con objetivos y visiones comunes. No representamos a nadie más que a nosotros mismos y a nuestro trabajo. Nuestro trabajo y dedicación frecuente a la libertad en todos los aspectos de la sociedad digital es lo que nos define.

Al crecer cada día más rápidamente la comunidad del Software Libre, se hace más importante mantener saludable, sólida y viva la visión del Software Libre. En algunos casos, el creciente interés político en las cuestiones que nos ocupan también crea el deseo de instrumentarlas para servir intereses personales particulares sin tener en cuenta los efectos a medio y largo plazo.

Por ello, el componente central de nuestro trabajo consiste en mantener fuerte, segura y libre de intereses particulares la base legal, política, y social del Software Libre. Ello requiere una comprensión profunda del Software Libre y aspectos relacionados. Por encima de todo requiere estar principalmente comprometido con la visión de largo plazo.

Comprendemos que a menudo esto significa tener que aceptar desventajas a corto o medio plazo para permanecer fieles a nuestros principios, que pueden ser duros de comunicar, y ocasionalmente incluso impopulares.

  \subsection{Principios}

Nos sentimos fuertemente comprometidos con los principios de democracia, transparencia, pluralismo, consistencia, confiabilidad y concentración.

A la luz de las sustanciales responsabilidades legales y de la necesidad de ser decisivos incluso en momentos difíciles, nuestra estructura pretende implantar los principios señalados lo mejor posible.
  \subsection{Estructura}

El trabajo y contribuciones voluntarios es lo fundamental sobre lo que descansa todo. Existen diferentes niveles de implicación; se trata principalmente de una decisión personal.

Todo el mundo es bienvenido a participar permanente, regular u ocasionalmente en las actividades de la Fundación del Software Libre de Europa (FSFE), y a hacer de las activities las suyas propias.

Si desea formar parte de la Fundación del Software Libre de Europa, puede unirse a uno de nuestros equipos, ya sea geográficamente o por tema, y tomar la responsabilidad de trabajar y actuar en nombre de la FSFE.

Los equipos basados en países se coordinan generalmente por los miembros de la asociación responsable de estos países (ver abajo). Los otros equipos se coordinan por miembros de la asociación, por miembros de los equipos o miembros de una organización asociada (ver abajo).

Si desea formar parte de la Fundación del Software Libre de Europa, pero no tiene tiempo para aportar trabajo, podría también considerar hacer una donación a la FSFE o ser un «fellow» de la FSFE.

Si su asociación tiene objetivos similares a los de la Fundación del Software Libre de Europa y desea establecer cooperación formal, puede obtener el estatus de organización asociada de la FSFE, llegando a ser de este modo parte de la red de la FSFE.

El esqueleto de la Fundación del Software Libre de Europa es asociacián formal y legalmente establecida con sus secciones en países miembros. Los miembros de esta asociación requieren el máximo compromiso, medido en años de dedicación y la responsabilidad de situar el consenso de largo plazo por delante de la opinión personal.

La asociación FSFE es fundamentalmente democrática. Todas las partes de la FSFE -los miembros de la asociación, los miembros de los equipos y los miembros de las organizaciones asociadas, así como los fellows- son bienvenidas a participar en el proceso de toma de decisiones.

Los miembros de la asociación son propuestos normalmente por miembros de los equipos y de las organizaciones asociadas de sus países. Son a continuación aprobadas por la asamblea general de la asociación FSFE.

Aunque las contribuciones voluntarias, en términos de trabajo y recursos, son los cimientos de nuestro trabajo, reconocemos que algunas de estas tareas no pueden realizarse como actividades a tiempo parcial y por lo tanto requieren trabajo a tiempo completo.

Los empleados normales de la Fundación del Software Libre de Europa realizan trabajos de producción y no forman parte de los procesos generales de toma de decisiones.

Cualquier decisión sobre contrataciones y pagos debe ser consensuada por la asamblea general de los miembros de la asociación.

El principal criterio para estas decisiones son las necesidades de la Fundación del Software Libre de Europa para cumplir sus objetivos con éxito. Siempre que es posible, intentamos contratar para cada trabajo a una persona que ya ha realizado el trabajo sin ser pagado, ya sea dentro o fuera de la FSFE, porque el conocimiento, la dedicación y la iniciativa son virtudes de la FSFE

Idealmente permitiremos que alguien se dedique completamente a una tarea que ya sea lo suficientemente importante para esa persona como para contribuir gran parte de sus recursos disponibles.

  \subsection{Procesos de decisión}

Las personas de la Fundación del Software Libre de Europa creemos en el consenso. Siempre buscamos basar nuestro trabajo en el consenso, y a veces en el compromiso, de nuestros miembros activos.

Además, creemos que es necesario en ocasiones realizar acciones rápidas y decisivas. Por esta razón hemos establecido la asociación FSFE y sus comités ejecutivos ampliados en niveles europeo y nacional. Éstos aportan estructuras y procedimientos de repliegue que se determinan, vigilan y controlan mediante procesos democráticos.

Este planteamiento se adoptó en la búsqueda de una estructura que permitiese la transparencia, pluralismo y participación, y al mismo tiempo que permaneciese tan sencilla como fuese posible.

Asegura la posibilidad de participación de todas las partes de FSFE. Los miembros de la asociación, los miembros de los equipos y los miembros de las organizaciones asociadas pueden todos participar. Esto permite actuar a la Fundación del Software Libre de Europa de forma rápida cuando es necesario, y mantener una fuerte consistencia de organización a largo plazo.

\section{\includegraphics{item.png}Grupos en la FSFE}

Existen varias formas de trabajar en equipo para un fin concreto, básicamente se trata de los \textit{equipos} y los \textit{grupos}

\begin{itemize}
\item Grupos Locales, donde la gente se reune localmente para organizar actividades y charlas sobre cuestiones relativas a la FSFE y/o el Software Libre. Cada grupo oficial FSFE tiene un coordinador que organiza los encuentros y suele ser la persona de contacto. Las reuniones de grupos locales están cubiertas por el \textsc{Código de Conducta}\footnote{https://fsfe.org/about/codeofconduct} de la FSFE 
\item Equipos Nacionales, formados por gente con el mismo idioma o cultura o pertenecientes a una misma administración territorial que se organizan en torno a una lista de correo y otras herramientas generalmente sin reuniones presenciales. Cada equipo nacional tiene un coordinador al que debes dirigirte si quieres unirte al equipo.
\item Equipos, enfocados en una materia concreta (licencias, android, traducciones, etc). A diferencia de los grupos locales las personas que los forman no tienen que estar necesariamente en la región en la que enfocan su trabajo, son grupos transversales en torno a un tema concreto. Tienen una área de trabajo definida dentro de la FSFE para colaborar con su misión. Tienen al menos un coordinador que asume la responsabilidad de acoger nuevos voluntarios y ejercer de enlace entre los diferentes equipos.
\end{itemize}

\section{\includegraphics{item.png}Grupo Local Noroeste}

Estamos intentando formar un grupo local de la FSFE con un ámbito geográfico aún por definir pero circunscrito al noroeste de la península ibérica.

La razón de formar este grupo es difundir el Software Libre y los derechos digitales, ayudar a la misión de la FSFE que consideramos muy necesaria especialmente en nuestro pais y contribuir a extender la presencia de la FSFE en toda Europa.

Los objetivos de actuación se enmarcaría en el plano de la difusión (organizando charlas y stands de la FSFE, participando en eventos, distribuyendo material, etc), la acción mediante campañas políticas con el objetivo a medio-largo de ejercer de lobby en las administraciones y organizaciones políticas españolas.

No se requiere un nivel técnico en informática para participar, al contrario todos los perfiles son necesarios y bienvenidos.

\vspace{5em}

\begin{mdframed}[style=MyFrame]
Queremos formar el \textsc{grupo local noroeste} de la FSFE para colaborar en los objetivos de la FSFE.

Si estás interesado en el Software Libre, en la privacidad, en los derechos de los usuarios de Software, en la transparencia, en la seguridad y en definitiva en los derechos digitales, anímate a participar y contactar con nosotros.

\center \Huge{¡Únete!}

\includegraphics[scale=0.8]{tw.png} \Large @pepdiz pd@fsfe.org
\end{mdframed}






\end{document}