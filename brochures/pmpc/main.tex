\documentclass[10pt,foldmark,tumble]{leaflet}
\usepackage[utf8]{inputenc}
\usepackage{mathptmx}
\usepackage{url}
\usepackage{graphicx,txfonts}
\usepackage[dvipsnames,usenames]{color}
\usepackage{dirtytalk}

\usepackage[framemethod=TikZ]{mdframed}
\mdfdefinestyle{MyFrame}{%
    linecolor=blue,
    outerlinewidth=2pt,
    roundcorner=20pt,
    innertopmargin=\baselineskip,
    innerbottommargin=\baselineskip,
    innerrightmargin=20pt,
    innerleftmargin=20pt,
    backgroundcolor=gray!30!white}

\newcommand{\love}{\ensuremath\varheartsuit\ }

\CutLine*{1}% Dotted line without scissors
\CutLine*{6}
%%\CutLine{6}%  Dotted line with scissors

%%\AddToBackground{6}{%  Background of a small page
%%  \put(0,0){\textcolor{Cyan}{\rule{\paperwidth}{\paperheight}}}}


\begin{document}

\includegraphics[scale=1.5]{pmpc.png}

\vspace{2em}

\begin{mdframed}[style=MyFrame]
\say{\textit{Queremos una legislación que permita que el software desarrollado para el
sector público y financiado con recursos públicos esté disponible públicamente
bajo una licencia de Software Libre y Código Abierto. Si es dinero público
debería ser también código público.
¡El código pagado por los ciudadanos debería estar disponible para los
ciudadanos!}}
\end{mdframed}

\vspace{2em}
¿Crees que el Software Libre debe ser la opción principal para el software financiado públicamente? Necesitamos convencer a los políticos que nos representan!

\vspace{3em}


\section{\includegraphics{item.png}Motivaciones para el código público}

\subsection{Ahorro de impuestos}

Aplicaciones similares no tienen que ser programadas desde cero una y otra vez.

\subsection{Colaboración}

Los esfuerzos en grandes proyectos se pueden compartir con lo que se gana experiencia y se reducen costos.

\subsection{Accesible a todos}

Las aplicaciones pagadas por los ciudadanos deben estar disponibles para todos.

\subsection{Estimular la innovación}

Con procesos transparentes los otros no tienen que reinventar la rueda.

\vspace{1em}

El software financiado con fondos públicos ha de ser Software Libre y de Código Abierto. Aunque hay muy buenas razones para ello, muchos políticos aún no lo saben.

\vspace{2em}

\begin{mdframed}[style=MyFrame]
\say{\textit{El Software Libre da a todos el derecho a usar, estudiar, compartir y mejorar el software. En estos derechos se fundan otras libertades fundamentales como la libertad de expresión, de prensa y el derecho a la privacidad}}
\end{mdframed}


\vspace{2em}

¡Ahí es donde puede ayudar! Firme la carta abierta para dar más peso a nuestro mensaje, como 14195 personas y 95 organizaciones hicieron antes que usted. Entregaremos la carta y las firmas a sus representantes y nos aseguraremos de que entiendan: ¿Dinero Público? ¡Código Público!



\section{\includegraphics{item.png}Carta Abierta}

En nuestra carta abierta pedimos:

\textit{Implementar una legislación que permita que el software desarrollado para el sector público y con dinero público esté disponible públicamente bajo una licencia de Software Libre y Código Abierto}

148 organizaciones y 17632 individuos ya están acompañando este llamado a participar, ellos están firmando nuestra carta abierta. ¡Firmándola tú nos puedes ayudar a conseguir un mayor impacto! Nosotros entregaremos todas las firmas a los representantes en toda Europa que están debatiendo sobre el software libre en la administración pública.

\vspace{1em}
  \subsection{ ¿Dinero público? ¡Código Público!}
\vspace{.5em}

    Los servicios digitales que ofrecen y utilizan nuestras administraciones públicas son la infraestructura crítica de las naciones democráticas del siglo XXI. Para establecer sistemas confiables, los organismos públicos deben asegurarse de que tienen control total sobre el software y los sistemas de computación en el centro de nuestra infraestructura digital estatal. Sin embargo, en este momento, esto es raro debido a las licencias de software restrictivas que:

\begin{itemize}
    \item Prohíben compartir e intercambiar código financiado con fondos públicos. Esto impide la cooperación entre las administraciones públicas y obstaculiza el desarrollo futuro.
    \item Apoyan los monopolios obstaculizando la competencia. Como resultado, muchas administraciones dependen de un puñado de empresas.
    \item Ponen en peligro la seguridad de nuestra infraestructura digital prohibiendo el acceso al código fuente. Esto hace que la localización de puertas traseras y agujeros de seguridad sea extremadamente difícil, si no imposible.
\end{itemize}

    Necesitamos software que fomente el intercambio de buenas ideas y soluciones. De este modo, podremos mejorar los servicios de TI para las personas de toda Europa. Necesitamos un software que garantice la libertad de elección, acceso y competencia. Necesitamos un software que ayude a las administraciones públicas a recuperar el control total de su infraestructura digital crítica, permitiéndoles independizarse y mantenerse independientes de un puñado de empresas. Por eso llamamos a nuestros representantes para que apoyen el Software Libre y de Código Abierto en las administraciones públicas, porque:

\begin{itemize}
    \item El Software Libre y de Código Abierto es un bien público moderno que permite a todos utilizar, estudiar, compartir y mejorar libremente las aplicaciones que utilizamos a diario.
    \item Las licencias de Software Libre y de Código Abierto ofrecen salvaguardias contra el bloqueo de los servicios de empresas específicas que utilizan licencias restrictivas para obstaculizar la competencia.
    \item Las licencias de Software Libre y de Código Abierto ofrecen salvaguardias contra el bloqueo de los servicios de empresas específicas que utilizan licencias restrictivas para obstaculizar la competencia.
\end{itemize}


    Los organismos públicos se financian a través de impuestos. Deben asegurarse de gastar los fondos de la manera más eficiente posible. Si se trata de dinero público, ¡también debe de ser código público!

    Por eso nosotros, los abajo firmantes, llamamos a nuestros representantes a

    \emph{Aplicar la legislación que exige que el software financiado con fondos públicos desarrollado para el sector público se ponga a disposición del público bajo una licencia de Software Libre y de Código Abierto}.


\newpage

\vspace{15em}


\includegraphics[scale=1.6]{pmpc.png} 

\vspace{1em}

\centering \includegraphics[scale=0.2]{fsfe.pdf}

\newpage
\section{\includegraphics{item.png}Organizaciones acompañantes}
\vspace{1em}
\includegraphics[scale=0.40]{colab.png} 


%%\centering \includegraphics[scale=0.45]{logo.pdf} \\
%%\centering \Huge \url{fsfe.org}


\end{document}
